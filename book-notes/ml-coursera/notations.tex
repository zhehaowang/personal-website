\documentclass{article}

\usepackage{amsmath}

\setlength\parindent{0pt}
\newcommand{\norm}[1]{\left\lVert#1\right\rVert}

\DeclareMathOperator*{\argmax}{arg\,max}
\DeclareMathOperator*{\mmax}{max}
\DeclareMathOperator*{\argmin}{arg\,min}

\begin{document}

\title{Defitions and notations}
\author{Zhehao Wang}

\maketitle{}

\subsection{Limit}

Let $f(x)$ be a function defined on an interval that contains $x = a$, except possibly at $x = a$, then we say that
$$
\lim\limits_{x \to a}{f(x) = L}
$$
if for every $\epsilon > 0$ there is some number $\delta > 0$ such that
$$
|f(x) - L| < \epsilon ~ \text{whenever} ~ 0 < |x - a| < \delta
$$

\subsection{Gradient}
Given $f(x_1, x_n)$ on $\mathbf{R}^n$
$$
\bigtriangledown f(a_1, ..., a_n) = (\frac{\partial}{\partial x_1}(a_1, ..., a_n), ..., \frac{\partial}{\partial x_n}(a_1, ..., a_n))
$$

\subsection{Directional derivative}

\textbf{Homework notation}
$$
f'(x; u) = \lim\limits_{h \to 0}{\frac{f(x + hu) - f(x)}{h}}
$$

\textbf{Wikipedias notation}
note that it's $f$ on a vector $\vec{x}$ having its derivative calculated on direction $\vec{v}$, essentially gradient on direction $\vec{v}$.
$$
\bigtriangledown_v f(\vec{x}) = \lim\limits_{h \to 0}{\frac{f(\vec{x} + h\vec{v}) - f(\vec{x})}{h}}
$$

\subsection{Vector norm}
A norm is a function that assigns a stricly positive length or size to each vector in a vector space (except the zero vector which is assigned a length of 0).

\textbf{Absolute value norm} is a norm on the one-dimensional vector spaces formed by real or complex numbers.
$$
\norm{x} = |x|
$$

\textbf{Euclidean norm} on a Euclidean space $\mathbf{R}^n$ is such
$$
\norm{\vec{x}}_2 = \sqrt{x_1^2 + ... + x_n^2}
$$

\textbf{Manhattan or taxicab norm}
$$
\norm{\vec{x}}_1 = \sum_{i = 1}^{n}{|x_i|}
$$

\textbf{$p$-norm}
$$
\norm{\vec{x}}_p = (\sum_{i = 1}^{n}{|x_i|^p})^{\frac{1}{p}}
$$
Note that when $p = 1$, we get Manhattan norm, and when $p = 2$, we get Euclidean norm.

When $p = \infty$
$$
\norm{\vec{x}}_{\infty} = {\mmax_{i}{|x_i|}}
$$

\subsection{argmax}

Points of the domain of some function at which the function values are maximized.

Given an arbitrary set $X$, a totally ordered set $Y$ and a function $f: X \to Y$, the $\argmax$ over some subset $S$ of $X$ is defined by
$$
\argmax_{x \in S \subseteq X}{f(x)} = \{x ~ | ~ x \in S \land \forall y \in S : f(y) \leq f(x)\}
$$


\end{document}